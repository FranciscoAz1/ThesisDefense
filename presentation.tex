%%%%%%%%%%%%%%%%%%%%%%%%%%%%% slides.tex %%%%%%%%%%%%%%%%%%%%%%%%%%%%%
%                                                                    %
%  Beamer template for slideshow presentations                       %
%                                                                    %
%       Andre C. Marta                                               %
%       Area Cientifica de Mecanica Aplicada e Aeroespacial          %
%       Departamento de Engenharia Mecanica                          %
%       Instituto Superior Tecnico                                   %
%       Av. Rovisco Pais                                             %
%       1049-001 Lisboa                                              %
%       Portugal                                                     %
%       Tel: +351 21 841 9469                                        %
%                        3469 (extension)                            %
%       Email: andre.marta@tecnico.ulisboa.pt                        %
%                                                                    %
%  Created:       Jan 12, 2023                                       %
%  Last Modified: Jan 12, 2023                                       %
%%%%%%%%%%%%%%%%%%%%%%%%%%%%%%%%%%%%%%%%%%%%%%%%%%%%%%%%%%%%%%%%%%%%%%
% This document uses:                                                %
% - LaTeX style "beamer.sty",                                        %
% - Beamer theme "beamerthemeIST.sty" (custom developed)             %
%%%%%%%%%%%%%%%%%%%%%%%%%%%%%%%%%%%%%%%%%%%%%%%%%%%%%%%%%%%%%%%%%%%%%%
% Beamer class options:                                              %
%                                                                    %
%   slidestop    - puts frame titles on the top left corner          %
%   compress     - makes all navigation bars as small as possible    %
%   blue,red,brown,baclandwhite - navigation bars and titles color   %
%   (default),handout, trans,notes=hide,show,only - pdf screen       %
%                 (128mmx96mm), pdf handout, pdf transparency, notes %
%   hyperref={bookmarks=false} - remove bookmarks                    %
%   (sans),sefif - text fonts                                        %
%   (mathsans), mathserif - math fonts                               %
%   8pt,9pt,10pt,(11pt),12pt,14pt,17pt,20pt - default font size      %
%   t,(c,)b      - vertical aligment for the entire document         %
%   draft        - show only figure place holders                    %
%%%%%%%%%%%%%%%%%%%%%%%%%%%%%%%%%%%%%%%%%%%%%%%%%%%%%%%%%%%%%%%%%%%%%%

\documentclass[]{beamer}
\mode<presentation> {
  % Beamer theme - user-definable (fonts,colors,decorations,...)
  \usetheme{IST} % beamerthemeIST.sty
  }

%%%%%%%%%%%%%%%%%%%%%%%%%%%%%%%%%%%%%%%%%%%%%%%%%%%%%%%%%%%%%%%%%%%%%%
% Load additional LaTeX packages

% Mathematical typesetting from the American Mathematical Society
\usepackage{amssymb,amsmath}

% Working with graphics with PDF LaTeX: using the graphicx package to
% incorporate graphics in pdf, jpg or png formats.
\usepackage{graphicx}  % Enhanced LaTeX Graphics

% Ac­cept dif­fer­ent in­put en­cod­ings
% http://www.ctan.org/pkg/inputenc
\usepackage[utf8]{inputenc} % supports portuguese keyboard in Linux

%% Access bold symbols in math mode. 
%% http://www.ctan.org/tex-archive/help/Catalogue/entries/bm.html
\usepackage{bm}

% Full bib­li­og­ra­phy en­tries in the main text of a document
% http://www.ctan.org/pkg/bibentry
\usepackage{bibentry}
\nobibliography*
\renewcommand{\newblock}{}

% Redefine footnotesize (used for bibliographic references)
\let\oldfootnotesize\footnotesize
\renewcommand*{\footnotesize}{\oldfootnotesize\tiny}

% Display a grid to help align images
%\beamertemplategridbackground[1cm]

% Display logo in frame title
\usepackage{textpos}
\addtobeamertemplate{frametitle}{}{%
\begin{textblock*}{100mm}(.9\textwidth,-8mm)
\includegraphics[height=7mm]{figures/IST_A_gray_bg_crop.jpg}
\end{textblock*}}

%\logo{\includegraphics[height=0.8cm]{figures/IST_A_gray_bg_crop.jpg}\vspace{200pt}}

%%%%%%%%%%%%%%%%%%%%%%%%%%%%%%%%%%%%%%%%%%%%%%%%%%%%%%%%%%%%%%%%%%%%%%
% Title, authors, affiliation and date
%%%%%%%%%%%%%%%%%%%%%%%%%%%%%%%%%%%%%%%%%%%%%%%%%%%%%%%%%%%%%%%%%%%%%%

\title[Enhancing Semantic Search]{Enhancing Semantic Search with Retrieval-Augmented Generation and Agentic AI}

\author[Francisco Azeredo]{Francisco Azeredo}

\titlegraphic{\includegraphics[height=8mm]{figures/IST_A_white_bg_crop.jpg}}

\institute[IST]{Instituto Superior T\'{e}cnico, Universidade de Lisboa}

\date[Thesis Defense]{MSc Thesis Defense}

%%%%%%%%%%%%%%%%%%%%%%%%%%%%%%%%%%%%%%%%%%%%%%%%%%%%%%%%%%%%%%%%%%%%%%
%%%%%%%%%%%%%%%%%%%%%%%%%%%%%%%%%%%%%%%%%%%%%%%%%%%%%%%%%%%%%%%%%%%%%%
\begin{document}

%%%%%%%%%%%%%%%%%%%%%%%%%%%%%%%%%%%%%%%%%%%%%%%%%%%%%%%%%%%%%%%%%%%%%%
% Frame options
%
%  plain            - supress drawing of header and footer decorations
%  containsverbatim - use verbatim environment and \verb command
%  allowframebreaks - automatic split of frames
%  shrink           - shrink the contents to fit in a single slide
%  squeeze          - squeeze vertical space
%  t,(c,)b          - top, center, bottom vertical alignment
%%%%%%%%%%%%%%%%%%%%%%%%%%%%%%%%%%%%%%%%%%%%%%%%%%%%%%%%%%%%%%%%%%%%%%

%%%%%%%%%%%%%%%%%%%%%%%%%%%%%%%%%%%%%%%%%%%%%%%%%%%%%%%%%%%%%%%%%%%%%%
% Title page
%%%%%%%%%%%%%%%%%%%%%%%%%%%%%%%%%%%%%%%%%%%%%%%%%%%%%%%%%%%%%%%%%%%%%%
\begin{frame}[plain] % supress drawing of header and footer decorations
  \titlepage
\end{frame}

% Reset page counter to exclude title page
\addtocounter{framenumber}{-1}

%%%%%%%%%%%%%%%%%%%%%%%%%%%%%%%%%%%%%%%%%%%%%%%%%%%%%%%%%%%%%%%%%%%%%%
% (No automatic outline frames; custom outline slide later)
%%%%%%%%%%%%%%%%%%%%%%%%%%%%%%%%%%%%%%%%%%%%%%%%%%%%%%%%%%%%%%%%%%%%%%

%--------------------------------------------------------------------
% INTRODUCTION
%--------------------------------------------------------------------
\section{Objectives \\ \& Outline}

\begin{frame}{Presentation Outline}
\begin{enumerate}
  \item Problem \& Motivation
  \item Objectives
  \item System Architecture
  \item Evaluation \& Results
  \item Conclusions \& Future Work
\end{enumerate}
\end{frame}

\begin{frame}{Research Problem}
\begin{block}{Core Challenge}
How to handle \textbf{interdependent information} in enterprise settings where information is \textbf{dynamic} and later reports can nullify earlier ones?
\end{block}
\vspace{0.5em}
\textbf{Key gaps in existing solutions:}
\begin{itemize}
  \item Keyword search ignores semantic meaning and document structure
  \item Naive RAG retrieves isolated information without considering dependencies
  \item Cannot track how information relates to, updates, or contradicts each other
  \item No standard tools for agents to traverse information relationships
\end{itemize}
\end{frame}



\section{Problem \\ \& Motivation}

\begin{frame}{Motivation: Why Corporate Search Fails}
\textbf{Traditional keyword search limitations:}
\begin{itemize}
  \item Cannot capture context, semantics, or document structure
  \item Users must guess exact terms and file locations
  \item Ignores relationships between information pieces
\end{itemize}
\vspace{0.5em}
\textbf{Naive RAG limitations in enterprise settings:}
\begin{itemize}
  \item Retrieves relevant information but \textbf{ignores interdependencies}
  \item Cannot detect when newer reports nullify or update older information
  \item Treats information store as static, missing \textbf{dynamic information flow}
  \item High risk: outdated or contradicted information in critical decisions
\end{itemize}
\end{frame}

\begin{frame}{The Document Interdependency Problem}
\textbf{Example scenario:}
\begin{itemize}
  \item \textbf{Report A (Jan 2024):} "Vendor X approved for contracts up to \$100K"
  \item \textbf{Report B (Mar 2024):} "Vendor X approval suspended due to audit"
  \item \textbf{Query (Apr 2024):} "Can we contract with Vendor X?"
\end{itemize}
\vspace{0.5em}
\textbf{Naive RAG problem:}
\begin{itemize}
  \item May retrieve Report A (highly relevant to query)
  \item Misses that Report B nullifies Report A
  \item Returns \textbf{accurate but outdated} information $\rightarrow$ incorrect answer
\end{itemize}
\vspace{0.3em}
\textbf{Solution needed:} Graph-based reasoning to traverse information relationships
\end{frame}
\begin{frame}{Thesis Objectives}
\textbf{Main goal:} Handle information interdependencies through graph-based agentic reasoning.
\vspace{0.5em}

\textbf{Specific objectives:}
\begin{enumerate}
  \item Design a schema that captures enterprise document structure and relationships
  \item Implement GraphRAG: knowledge graph construction from document corpus
  \item Develop agentic graph traversal for multi-hop reasoning over dependencies
  \item Create an MCP server for standardized agent-database interaction
  \item Evaluate against naive RAG on enterprise-like scenarios
\end{enumerate}
\end{frame}
\section{Architecture}

\begin{frame}{System Architecture Overview}
\textbf{Three-layer architecture:}
\begin{enumerate}
  \item \textbf{Processing:} OCR, chunking, embeddings, metadata extraction
  \item \textbf{Storage:} Vector database (Weaviate) + Knowledge graph for information relationships
  \item \textbf{Reasoning:} ReAct agent with schema-aware tools (MCP)
\end{enumerate}
\begin{center}
  \includegraphics[width=0.75\linewidth]{Images/Fluxograma_Data_Processing_Pipeline.jpeg}
\end{center}
\end{frame}

\begin{frame}{Schema-Aware Data Model}
\textbf{Six core classes model enterprise document structure:}
\begin{itemize}
  \item \textbf{Fluxo} (Workflow), \textbf{Etapa} (Step): process structure
  \item \textbf{Entidade} (Entity): companies, people, products
  \item \textbf{Pasta} (Folder), \textbf{Ficheiro} (File), \textbf{Metadados} (Metadata)
\end{itemize}
\vspace{0.3em}
\textbf{Key benefit:} Cross-references enable both semantic search \textit{and} deterministic multi-hop traversal.
\begin{center}
  \includegraphics[width=0.7\linewidth]{Images/Classe UML.png}
\end{center}
\end{frame}

\begin{frame}{Knowledge Graph Construction (MiniRAG)}
\textbf{Lightweight GraphRAG implementation:}
\begin{itemize}
  \item LLM extracts key entity words from document chunks
  \item Nodes with matching entity words are merged
  \item Creates graph structure connecting related information
  \item Requires supervision but aids in processing large document sets
\end{itemize}
\vspace{0.3em}
\textbf{Goal:} Facilitate discovery of information relationships across documents through entity-based graph structure.
\begin{center}
  \includegraphics[width=0.5\linewidth]{Images/graph_visualisation_big.png}
\end{center}
\end{frame}

\begin{frame}{Agentic Graph Traversal (Mini Query Mode)}
\textbf{Multi-hop reasoning over knowledge graph:}
\begin{enumerate}
  \item Extract entities and temporal context from query
  \item Retrieve relevant information via vector search
  \item \textbf{Traverse graph} to find related/updating information
  \item Build context considering dependencies and temporal order
  \item Generate answer from complete, up-to-date evidence
\end{enumerate}
\begin{center}
  \includegraphics[width=0.65\linewidth]{Images/Fluxograma_Mini_Query.jpeg}
\end{center}
\end{frame}

\begin{frame}{Weaviate MCP Server: Agent-Database Bridge}
\textbf{Model Context Protocol (MCP) standardizes agent tool interfaces.}

\textbf{Our contribution:} First MCP server for Weaviate
\begin{itemize}
  \item Exposes tools: \texttt{weaviate-query}, \texttt{weaviate-follow-ref}, etc.
\end{itemize}
\begin{center}
  \includegraphics[width=0.6\linewidth]{Images/Sequence-diagram-Weaviate-Query.png}
\end{center}
\end{frame}

\section{Results}

\begin{frame}{Results: Retrieval Performance}
\textbf{Experiment:} Single-collection retrieval on enterprise-like dataset
\vspace{0.5em}

\textbf{Key findings:}
\begin{itemize}
  \item \textbf{Agentic RAG}: 60.4\% retrieval rate, 61\% correct answers
  \item Outperforms all baselines by large margin (2-4×)
  \item Agentic approach combines multi-step reasoning with hybrid search
\end{itemize}
\begin{center}
  % Single-collection retrieval and answer quality
  % Auto-generated metrics table
\begin{table}[t]
\centering
\caption{Document retrieval and answer quality metrics}
\label{tab:document_retrieved_metrics}
\begin{tabular}{lrr}
\hline
Method & Avg. Retrieval Rate & Correct Answers \\
\hline
Agentic RAG & 60.4\% & 61\% \\
LexRank & 33.6\% & 29.5\% \\
BART & 17.8\% & 3.0\% \\
Naive RAG & 13.8\% & 19.1\% \\
\hline
\end{tabular}
\end{table}

\end{center}
\footnotesize{``Correct Answers'' = answers passing Youden-optimized quality threshold}
\end{frame}

\begin{frame}{Results: Multi-Collection Retrieval}
\textbf{Experiment:} Retrieval across heterogeneous document collections
\vspace{0.5em}

\textbf{Key findings:}
\begin{itemize}
  \item Agent successfully selects appropriate collection for each query
  \item All agentic methods outperform naive baseline (2-3×)
  \item Different strategies (REST API, LexRank, DOCX) suit different content
\end{itemize}
\begin{center}
  % Mixed-documents retrieval metrics
  % Auto-generated metrics table
\begin{table}[t]
\centering
\caption{Multi-collection retrieval and answer quality metrics}
\label{tab:document_retrieved_metrics}
\begin{tabular}{lrr}
\hline
Method & Avg. Retrieval Rate & Correct Answers \\
\hline
Mixed REST & 46.5\% & 15.8\% \\
Mixed LexRank & 39.4\% & 24.6\% \\
Mixed DOCX & 33.0\% & 31.5\% \\
Naive RAG & 13.8\% & 19.1\% \\
\hline
\end{tabular}
\end{table}

\end{center}
\footnotesize{Demonstrates generalization to diverse document types and sources}
\end{frame}

\section{Conclusions \& Future Work}

\begin{frame}{Main Contributions}
\begin{enumerate}
  \item \textbf{Problem identification:} Formalized the information interdependency challenge in enterprise semantic search
  \item \textbf{GraphRAG architecture:} Schema-aware system capturing information relationships, updates, and temporal dependencies
  \item \textbf{Agentic graph traversal:} Multi-hop reasoning framework to handle dynamic information and nullification
  \item \textbf{MCP server:} First standardized tool for agents to query Weaviate with schema validation
  \item \textbf{Evaluation:} Demonstrated 2-4× improvement over naive RAG on enterprise scenarios
\end{enumerate}
\end{frame}

\begin{frame}{Conclusions}
\begin{itemize}
  \item \textbf{GraphRAG with agentic traversal} handles information interdependencies that naive RAG cannot (60\% vs 14-34\% retrieval accuracy)
  \item \textbf{Multi-hop reasoning} over knowledge graphs enables tracking how information updates, supersedes, or contradicts earlier information
  \item Successfully addresses \textbf{dynamic information storage} problem in enterprise settings
  \item MCP server provides standardized, validated access to Weaviate for any LLM agent
  \item Demonstrates importance of \textbf{information structure and relationships} beyond semantic similarity
\end{itemize}
\end{frame}

\begin{frame}{Limitations \& Challenges}
\textbf{Experimental limitations:}
\begin{itemize}
  \item Computational and budget constraints limited scale of experiments
  \item No public benchmark exists for structure-aware enterprise retrieval
\end{itemize}
\vspace{0.5em}
\textbf{Open challenges:}
\begin{itemize}
  \item LLM hallucinations not fully eliminated (though greatly reduced)
  \item Privacy and security considerations in enterprise deployment
  \item Scalability to millions of information pieces requires further optimization
\end{itemize}
\end{frame}

\begin{frame}{Future Work}
\begin{itemize}
  \item Evaluate stronger models and longer context windows.
  \item Develop public benchmarks for structure-aware retrieval.
  \item Explore advanced reasoning (multi-agent, graph reasoning, GraphRAG).
  \item Integrate with enterprise systems and enforce ACLs at scale.
  \item Scale to millions of information pieces with optimized graph traversal.
\end{itemize}
\end{frame}

\begin{frame}{Thank You}
\begin{center}
  \begin{block}{\LARGE Thank you!}
  \end{block}
  \vspace{1em}
  Francisco Azeredo\\
  Instituto Superior T\'ecnico\\
  \vspace{0.5em}
  Questions?
\end{center}
\end{frame}

\end{document}
%%%%%%%%%%%%%%%%%%%%%%%%%%%%%%%%%%%%%%%%%%%%%%%%%%%%%%%%%%%%%%%%%%%%%%

