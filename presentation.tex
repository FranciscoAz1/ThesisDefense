%%%%%%%%%%%%%%%%%%%%%%%%%%%%% slides.tex %%%%%%%%%%%%%%%%%%%%%%%%%%%%%
%                                                                    %
%  Beamer template for slideshow presentations                       %
%                                                                    %
%       Andre C. Marta                                               %
%       Area Cientifica de Mecanica Aplicada e Aeroespacial          %
%       Departamento de Engenharia Mecanica                          %
%       Instituto Superior Tecnico                                   %
%       Av. Rovisco Pais                                             %
%       1049-001 Lisboa                                              %
%       Portugal                                                     %
%       Tel: +351 21 841 9469                                        %
%                        3469 (extension)                            %
%       Email: andre.marta@tecnico.ulisboa.pt                        %
%                                                                    %
%  Created:       Jan 12, 2023                                       %
%  Last Modified: Jan 12, 2023                                       %
%%%%%%%%%%%%%%%%%%%%%%%%%%%%%%%%%%%%%%%%%%%%%%%%%%%%%%%%%%%%%%%%%%%%%%
% This document uses:                                                %
% - LaTeX style "beamer.sty",                                        %
% - Beamer theme "beamerthemeIST.sty" (custom developed)             %
%%%%%%%%%%%%%%%%%%%%%%%%%%%%%%%%%%%%%%%%%%%%%%%%%%%%%%%%%%%%%%%%%%%%%%
% Beamer class options:                                              %
%                                                                    %
%   slidestop    - puts frame titles on the top left corner          %
%   compress     - makes all navigation bars as small as possible    %
%   blue,red,brown,baclandwhite - navigation bars and titles color   %
%   (default),handout, trans,notes=hide,show,only - pdf screen       %
%                 (128mmx96mm), pdf handout, pdf transparency, notes %
%   hyperref={bookmarks=false} - remove bookmarks                    %
%   (sans),sefif - text fonts                                        %
%   (mathsans), mathserif - math fonts                               %
%   8pt,9pt,10pt,(11pt),12pt,14pt,17pt,20pt - default font size      %
%   t,(c,)b      - vertical aligment for the entire document         %
%   draft        - show only figure place holders                    %
%%%%%%%%%%%%%%%%%%%%%%%%%%%%%%%%%%%%%%%%%%%%%%%%%%%%%%%%%%%%%%%%%%%%%%

\documentclass[]{beamer}
\mode<presentation> {
  % Beamer theme - user-definable (fonts,colors,decorations,...)
  \usetheme{IST} % beamerthemeIST.sty
  }

%%%%%%%%%%%%%%%%%%%%%%%%%%%%%%%%%%%%%%%%%%%%%%%%%%%%%%%%%%%%%%%%%%%%%%
% Load additional LaTeX packages

% Mathematical typesetting from the American Mathematical Society
\usepackage{amssymb,amsmath}

% Working with graphics with PDF LaTeX: using the graphicx package to
% incorporate graphics in pdf, jpg or png formats.
\usepackage{graphicx}  % Enhanced LaTeX Graphics

% Ac­cept dif­fer­ent in­put en­cod­ings
% http://www.ctan.org/pkg/inputenc
\usepackage[utf8]{inputenc} % supports portuguese keyboard in Linux

%% Access bold symbols in math mode. 
%% http://www.ctan.org/tex-archive/help/Catalogue/entries/bm.html
\usepackage{bm}

% Redefine footnotesize (used for bibliographic references)
\let\oldfootnotesize\footnotesize
\renewcommand*{\footnotesize}{\oldfootnotesize\tiny}

% Display a grid to help align images
%\beamertemplategridbackground[1cm]

% Display logo in frame title
\usepackage{textpos}
\addtobeamertemplate{frametitle}{}{%
\begin{textblock*}{100mm}(.9\textwidth,-8mm)
\includegraphics[height=7mm]{figures/IST_A_gray_bg_crop.jpg}
\end{textblock*}}

%\logo{\includegraphics[height=0.8cm]{figures/IST_A_gray_bg_crop.jpg}\vspace{200pt}}

%%%%%%%%%%%%%%%%%%%%%%%%%%%%%%%%%%%%%%%%%%%%%%%%%%%%%%%%%%%%%%%%%%%%%%
% Title, authors, affiliation and date
%%%%%%%%%%%%%%%%%%%%%%%%%%%%%%%%%%%%%%%%%%%%%%%%%%%%%%%%%%%%%%%%%%%%%%

\title[Enhancing Semantic Search]{Enhancing Semantic Search with Retrieval-Augmented Generation and Agentic AI}

% Add advisors (replace placeholders with real names)
% Updated with actual advisor names
\author[Francisco Azeredo]{\texorpdfstring{Francisco Azeredo\\\small Academic Advisor: Prof. Sérgio Luís Proença Duarte Guerreiro\\\small Industrial Advisor: Eng.º Filipe Mendes Correia (Link Consulting SA)}{Francisco Azeredo, Academic Advisor: Prof. Sérgio Luís Proença Duarte Guerreiro, Industrial Advisor: Eng.º Filipe Mendes Correia (Link Consulting SA)}}

\titlegraphic{\includegraphics[height=8mm]{figures/IST_A_white_bg_crop.jpg}}

\institute[IST]{Instituto Superior T\'{e}cnico, Universidade de Lisboa}

\date[Thesis Defense]{MSc Thesis Defense}

%%%%%%%%%%%%%%%%%%%%%%%%%%%%%%%%%%%%%%%%%%%%%%%%%%%%%%%%%%%%%%%%%%%%%%
%%%%%%%%%%%%%%%%%%%%%%%%%%%%%%%%%%%%%%%%%%%%%%%%%%%%%%%%%%%%%%%%%%%%%%
\begin{document}

%%%%%%%%%%%%%%%%%%%%%%%%%%%%%%%%%%%%%%%%%%%%%%%%%%%%%%%%%%%%%%%%%%%%%%
% Frame options
%
%  plain            - supress drawing of header and footer decorations
%  containsverbatim - use verbatim environment and \verb command
%  allowframebreaks - automatic split of frames
%  shrink           - shrink the contents to fit in a single slide
%  squeeze          - squeeze vertical space
%  t,(c,)b          - top, center, bottom vertical alignment
%%%%%%%%%%%%%%%%%%%%%%%%%%%%%%%%%%%%%%%%%%%%%%%%%%%%%%%%%%%%%%%%%%%%%%

%%%%%%%%%%%%%%%%%%%%%%%%%%%%%%%%%%%%%%%%%%%%%%%%%%%%%%%%%%%%%%%%%%%%%%
% Title page
%%%%%%%%%%%%%%%%%%%%%%%%%%%%%%%%%%%%%%%%%%%%%%%%%%%%%%%%%%%%%%%%%%%%%%
\begin{frame}[plain] % supress drawing of header and footer decorations
  	\titlepage
\end{frame}

% Reset page counter to exclude title page
\addtocounter{framenumber}{-1}

%%%%%%%%%%%%%%%%%%%%%%%%%%%%%%%%%%%%%%%%%%%%%%%%%%%%%%%%%%%%%%%%%%%%%%
% (No automatic outline frames; custom outline slide later)
%%%%%%%%%%%%%%%%%%%%%%%%%%%%%%%%%%%%%%%%%%%%%%%%%%%%%%%%%%%%%%%%%%%%%%

%--------------------------------------------------------------------
% INTRODUCTION
%--------------------------------------------------------------------
\section{Outline}

\begin{frame}{Presentation Structure}
\begin{enumerate}
  \item Motivation and Problem
  \item Objectives
  \item Architecture and Components
  \item Related Work (Context)
  \item Evaluation and Results
  \item Conclusions and Future Work
\end{enumerate}
\end{frame}

\section{Motivation and Problem}

\begin{frame}{Enterprise Information Challenges}
\textbf{Critical Business Context:}
\begin{itemize}
  \item Enterprises rely on \textit{accurate, current} information for project approvals, compliance decisions, and operational governance
  \item \textbf{High-stakes work}: Incorrect or outdated information can lead to regulatory violations, project failures, or financial losses
\end{itemize}
\vspace{0.3em}
\textbf{Current Problems:}
\begin{itemize}
  \item Professionals spend significant time searching, reading, and analyzing multiple documents
  \item Manual verification of document dependencies creates \textbf{high cognitive load}
  \item Decisions risk relying on superseded or nullified information
  \item \textbf{Even modern semantic search systems fail} in critical situations—they cannot guarantee information currency or detect contradictions
\end{itemize}
\end{frame}

\begin{frame}{Context: Edoclink Enterprise}
\textbf{Document Management Platform:}
\begin{itemize}
  \item Organizes documents with structured workflows and business rules
  \item Supports evolution from ad-hoc to complex workflow configurations
  \item Enables end-to-end process automation and digitalization
  \item Features: document lifecycle management, collaborative work, ERP integration
  \item Used in public sector and enterprises with rich, complex processes
\end{itemize}
\vspace{0.4em}
\textbf{Thesis Opportunity:} Leverage this rich structure to enhance semantic search and retrieval accuracy.
\end{frame}
\begin{frame}{Problem Illustration}
    put this image
    https://www.bbc.com/news/war-in-ukraine
\end{frame}

\begin{frame}{Problem: Example Scenario}
	\textbf{Example Timeline}
\begin{itemize}
  \item Jan: Russia declared war on Ukraine (report A)
  \item Mar: Russia and Ukraine agreed on a ceasefire (doc B)
  \item Apr Query: ``How is the war in Ukraine?'' % chktex 38 allow
\end{itemize}
	\textbf{Risk:} Retrieves A, ignores B..
\vspace{0.4em} \\
	\textbf{Core Challenge:} Capture and reason over document dependencies.
\end{frame}

\begin{frame}{Problem}
\begin{itemize}
  \item Current search methods don't account for information interdependencies
  \item GraphRAG ideas emerging; enterprise schemas underused
\end{itemize}
\end{frame}

\section{Objectives}

\begin{frame}{Objectives Overview}
\begin{itemize}
  \item \textbf{Represent} enterprise document structure and relations
  \item \textbf{Detect} updates / contradictions across documents
  \item \textbf{Enterprise Ready:}
  \begin{itemize}
      \item \textbf{Standardize} easy to implement solution in current emerging frameworks
      \item \textbf{Scalability} Work with highly scalable repositories of information
  \end{itemize}
  \item \textbf{Evaluate}: Evaluate efficiency, and quality of solutions
\end{itemize}
\end{frame}
%% Possibly remove
\begin{frame}{Objective Details}
\begin{enumerate}
  \item Schema design (workflows, entities, metadata)
  \item Graph construction (entity merging, references)
  \item Agent traversal (multi-hop, temporal ordering)
  \item MCP tools (validated queries, follow references)
  \item Evaluation (retrieval accuracy, answer quality)
\end{enumerate}
\end{frame}

\section{Architecture and Components}
\begin{frame}{Architecture Overview}
We developed 3 independent components for each step in document processing
\begin{enumerate}
  \item Insertion/Index (Automatic Knowledge Graph construction)
  \item Query (Query techniques, some that take better advantage of the graph)
  \item Generation (Context given to Agents or LLMs for a readable output)
\end{enumerate}
\vspace{0.3em}
\end{frame}

\begin{frame}{Knowledge graph construction}
\begin{enumerate}
  \item Extract entities and relationships
  \item Merge nodes with matching entities
  \item Store in a graph
\end{enumerate}
\vspace{0.3em}
\centering
\includegraphics[width=0.8\linewidth]{Images/Fluxograma_Data_Processing_Pipeline.jpeg}
\end{frame}

\begin{frame}{Query Techniques}
The query techniques go from the one of least latency to highest latency
\begin{enumerate}
    \item Semantic, bm25 and hybrid queries
    \item query reformulation (Extract entities and relationships from query) for graph retrieval
    \item ReAct (Agent) queries using MCP server
\end{enumerate}
\centering
\includegraphics[width=0.6\linewidth]{Images/Fluxograma_Mini_Query.jpeg}
\end{frame}

\begin{frame}{Agent Configuration}
\begin{enumerate}
  \item Instructed to be a document assistant, connected to a weaviate database, where it should base all of it's answers. And answer in a concise way with sources of information.
  \item Tools to a MCP server, that handles communication from the Agent (natural language) to the weaviate server (graphQL)
\end{enumerate}
\end{frame}

\begin{frame}{MCP Server (Weaviate Bridge)}
Valuable communications
\begin{itemize}
  \item Standard tool interface for agents
  \item Validated schema-aware queries
  \item Tools: weaviate-query, weaviate-origin, weaviate-follow-ref
\end{itemize}
weviate-origin appends the weaviate object to the LLM with references of that object, described by the relationship
weaviate-follow-ref is used for querying by following references of an object, and then doing the same weaviate-origin. Enabling agentic graph traversal.
\end{frame}

\section{Evaluation and Results}

\begin{frame}{Evaluation Questions}
\begin{itemize}
  \item Does graph traversal improve retrieval accuracy?
  \item Does dependency handling improve answer correctness?
  \item Does hybrid search generalize across collections?
\end{itemize}
\end{frame}

\begin{frame}{Single-Collection Performance}
\textbf{Agentic RAG}: 60.4\% retrieval, 61\% correct answers ($\sim$2--4$\times$ over baselines)
\vspace{0.3em}
\begin{center}
  % Auto-generated metrics table
\begin{table}[t]
\centering
\caption{Document retrieval and answer quality metrics}
\label{tab:document_retrieved_metrics}
\begin{tabular}{lrr}
\hline
Method & Avg. Retrieval Rate & Correct Answers \\
\hline
Agentic RAG & 60.4\% & 61\% \\
LexRank & 33.6\% & 29.5\% \\
BART & 17.8\% & 3.0\% \\
Naive RAG & 13.8\% & 19.1\% \\
\hline
\end{tabular}
\end{table}

\end{center}
\footnotesize Threshold via Youden optimization.
\end{frame}

\begin{frame}{Multi-Collection Retrieval}
\textbf{Generalization across heterogeneous sources}
\begin{itemize}
  \item Correct collection selection
  \item 2--3$\times$ improvement across strategies
  \item Robust to format variation
\end{itemize}
\begin{center}
  % Auto-generated metrics table
\begin{table}[t]
\centering
\caption{Multi-collection retrieval and answer quality metrics}
\label{tab:document_retrieved_metrics}
\begin{tabular}{lrr}
\hline
Method & Avg. Retrieval Rate & Correct Answers \\
\hline
Mixed REST & 46.5\% & 15.8\% \\
Mixed LexRank & 39.4\% & 24.6\% \\
Mixed DOCX & 33.0\% & 31.5\% \\
Naive RAG & 13.8\% & 19.1\% \\
\hline
\end{tabular}
\end{table}

\end{center}
\end{frame}

\begin{frame}{Key Result Insights}
\begin{itemize}
  \item Multi-hop reasoning reduces outdated answers
  \item Temporal ordering prevents contradiction leakage
  \item Graph adds structured context beyond embeddings
\end{itemize}
\end{frame}

\section{Related Work}
\begin{frame}{Related Work: Semantic and Graph RAG}
\begin{itemize}
  \item Keyword enterprise search: lacks semantics/structure
  \item Classic RAG: isolated chunk retrieval
  \item Emerging GraphRAG: entity/relationship surfacing
  \item Gap: temporal nullification + standardized tooling
\end{itemize}
\end{frame}

\begin{frame}{Comparison to Prior Approaches}
\begin{itemize}
  \item Adds temporal precedence handling
  \item Integrates schema-driven traversal
  \item Standardizes graph access (MCP server)
  \item Focus: enterprise interdependency reliability
\end{itemize}
\end{frame}
\section{Conclusions and Future Work}

\begin{frame}{Main Contributions}
\begin{enumerate}
  \item Formalized enterprise interdependency retrieval problem
  \item Schema + graph architecture for dynamic updates
  \item Agentic multi-hop traversal (temporal aware)
  \item MCP Weaviate server (standardized tools)
  \item Empirical gains (2–4× over baselines)
\end{enumerate}
\end{frame}

\begin{frame}{Conclusions}
\begin{itemize}
  \item \textbf{Graph-aware reasoning}: prevents outdated answers\pause
  \item \textbf{Temporal ordering}: resolves contradictions\pause
  \item \textbf{Schema constraints}: reduce hallucination surface\pause
  \item \textbf{Tool standardization} enables reproducible agent pipelines
\end{itemize}
\end{frame}

\begin{frame}{Limitations}
\begin{itemize}
  \item Scale constrained (compute / budget)
  \item No public benchmark (construction overhead)
  \item Residual hallucinations possible
  \item Need deeper privacy/ACL integration
\end{itemize}
\end{frame}

\begin{frame}{Future Work}
\begin{itemize}
  \item Larger-scale graph + incremental updates
  \item Benchmark release (structure-aware retrieval)
  \item Multi-agent verification and consistency checks
  \item Integration with access control layers
  \item Publication submission (CONF NAME / JOURNAL TBD)
\end{itemize}
\end{frame}

\begin{frame}{Thank You}
\begin{center}
  \begin{block}{\LARGE Thank you!}
  \end{block}
  \vspace{1em}
  Francisco Azeredo\\
  Instituto Superior T\'ecnico\\
  Questions?
\end{center}
\end{frame}

\end{document}
%%%%%%%%%%%%%%%%%%%%%%%%%%%%%%%%%%%%%%%%%%%%%%%%%%%%%%%%%%%%%%%%%%%%%%

