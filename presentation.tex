%%%%%%%%%%%%%%%%%%%%%%%%%%%%% slides.tex %%%%%%%%%%%%%%%%%%%%%%%%%%%%%
%                                                                    %
%  Beamer template for slideshow presentations                       %
%                                                                    %
%       Andre C. Marta                                               %
%       Area Cientifica de Mecanica Aplicada e Aeroespacial          %
%       Departamento de Engenharia Mecanica                          %
%       Instituto Superior Tecnico                                   %
%       Av. Rovisco Pais                                             %
%       1049-001 Lisboa                                              %
%       Portugal                                                     %
%       Tel: +351 21 841 9469                                        %
%                        3469 (extension)                            %
%       Email: andre.marta@tecnico.ulisboa.pt                        %
%                                                                    %
%  Created:       Jan 12, 2023                                       %
%  Last Modified: Jan 12, 2023                                       %
%%%%%%%%%%%%%%%%%%%%%%%%%%%%%%%%%%%%%%%%%%%%%%%%%%%%%%%%%%%%%%%%%%%%%%
% This document uses:                                                %
% - LaTeX style "beamer.sty",                                        %
% - Beamer theme "beamerthemeIST.sty" (custom developed)             %
%%%%%%%%%%%%%%%%%%%%%%%%%%%%%%%%%%%%%%%%%%%%%%%%%%%%%%%%%%%%%%%%%%%%%%
% Beamer class options:                                              %
%                                                                    %
%   slidestop    - puts frame titles on the top left corner          %
%   compress     - makes all navigation bars as small as possible    %
%   blue,red,brown,baclandwhite - navigation bars and titles color   %
%   (default),handout, trans,notes=hide,show,only - pdf screen       %
%                 (128mmx96mm), pdf handout, pdf transparency, notes %
%   hyperref={bookmarks=false} - remove bookmarks                    %
%   (sans),sefif - text fonts                                        %
%   (mathsans), mathserif - math fonts                               %
%   8pt,9pt,10pt,(11pt),12pt,14pt,17pt,20pt - default font size      %
%   t,(c,)b      - vertical aligment for the entire document         %
%   draft        - show only figure place holders                    %
%%%%%%%%%%%%%%%%%%%%%%%%%%%%%%%%%%%%%%%%%%%%%%%%%%%%%%%%%%%%%%%%%%%%%%

\documentclass[]{beamer}
\mode<presentation> {
  % Beamer theme - user-definable (fonts,colors,decorations,...)
  \usetheme{IST} % beamerthemeIST.sty
  }

%%%%%%%%%%%%%%%%%%%%%%%%%%%%%%%%%%%%%%%%%%%%%%%%%%%%%%%%%%%%%%%%%%%%%%
% Load additional LaTeX packages

% Mathematical typesetting from the American Mathematical Society
\usepackage{amssymb,amsmath}

% Working with graphics with PDF LaTeX: using the graphicx package to
% incorporate graphics in pdf, jpg or png formats.
\usepackage{graphicx}  % Enhanced LaTeX Graphics

% Ac­cept dif­fer­ent in­put en­cod­ings
% http://www.ctan.org/pkg/inputenc
\usepackage[utf8]{inputenc} % supports portuguese keyboard in Linux

%% Access bold symbols in math mode. 
%% http://www.ctan.org/tex-archive/help/Catalogue/entries/bm.html
\usepackage{bm}

% Redefine footnotesize (used for bibliographic references)
\let\oldfootnotesize\footnotesize
\renewcommand*{\footnotesize}{\oldfootnotesize\tiny}

% Display a grid to help align images
%\beamertemplategridbackground[1cm]

% Display logo in frame title
\usepackage{textpos}
\addtobeamertemplate{frametitle}{}{%
\begin{textblock*}{100mm}(.9\textwidth,-8mm)
\includegraphics[height=7mm]{figures/IST_A_gray_bg_crop.jpg}
\end{textblock*}}

%\logo{\includegraphics[height=0.8cm]{figures/IST_A_gray_bg_crop.jpg}\vspace{200pt}}

%%%%%%%%%%%%%%%%%%%%%%%%%%%%%%%%%%%%%%%%%%%%%%%%%%%%%%%%%%%%%%%%%%%%%%
% Title, authors, affiliation and date
%%%%%%%%%%%%%%%%%%%%%%%%%%%%%%%%%%%%%%%%%%%%%%%%%%%%%%%%%%%%%%%%%%%%%%

\title[Enhancing Semantic Search]{Enhancing Semantic Search with Retrieval-Augmented Generation and Agentic AI}

% Add advisors (replace placeholders with real names)
% Updated with actual advisor names
\author[Francisco Azeredo]{\texorpdfstring{Francisco Azeredo\\\small Academic Advisor: Prof. Sérgio Luís Proença Duarte Guerreiro\\\small Industrial Advisor: Eng.º Filipe Mendes Correia (Link Consulting SA)}{Francisco Azeredo, Academic Advisor: Prof. Sérgio Luís Proença Duarte Guerreiro, Industrial Advisor: Eng.º Filipe Mendes Correia (Link Consulting SA)}}

\titlegraphic{\includegraphics[height=8mm]{figures/IST_A_white_bg_crop.jpg}}

\institute[IST]{Instituto Superior T\'{e}cnico, Universidade de Lisboa}

\date[Thesis Defense]{MSc Thesis Defense}

%%%%%%%%%%%%%%%%%%%%%%%%%%%%%%%%%%%%%%%%%%%%%%%%%%%%%%%%%%%%%%%%%%%%%%
%%%%%%%%%%%%%%%%%%%%%%%%%%%%%%%%%%%%%%%%%%%%%%%%%%%%%%%%%%%%%%%%%%%%%%
\begin{document}

%%%%%%%%%%%%%%%%%%%%%%%%%%%%%%%%%%%%%%%%%%%%%%%%%%%%%%%%%%%%%%%%%%%%%%
% Frame options
%
%  plain            - supress drawing of header and footer decorations
%  containsverbatim - use verbatim environment and \verb command
%  allowframebreaks - automatic split of frames
%  shrink           - shrink the contents to fit in a single slide
%  squeeze          - squeeze vertical space
%  t,(c,)b          - top, center, bottom vertical alignment
%%%%%%%%%%%%%%%%%%%%%%%%%%%%%%%%%%%%%%%%%%%%%%%%%%%%%%%%%%%%%%%%%%%%%%

%%%%%%%%%%%%%%%%%%%%%%%%%%%%%%%%%%%%%%%%%%%%%%%%%%%%%%%%%%%%%%%%%%%%%%
% Title page
%%%%%%%%%%%%%%%%%%%%%%%%%%%%%%%%%%%%%%%%%%%%%%%%%%%%%%%%%%%%%%%%%%%%%%
\begin{frame}[plain] % supress drawing of header and footer decorations
  	\titlepage
\end{frame}

% Reset page counter to exclude title page
\addtocounter{framenumber}{-1}

%%%%%%%%%%%%%%%%%%%%%%%%%%%%%%%%%%%%%%%%%%%%%%%%%%%%%%%%%%%%%%%%%%%%%%
% (No automatic outline frames; custom outline slide later)
%%%%%%%%%%%%%%%%%%%%%%%%%%%%%%%%%%%%%%%%%%%%%%%%%%%%%%%%%%%%%%%%%%%%%%

%--------------------------------------------------------------------
% INTRODUCTION
%--------------------------------------------------------------------
\section{Outline}

\begin{frame}{Presentation Structure}
\begin{enumerate}
  \item Motivation and Problem
  \item Objectives
  \item Thesis Work
  \begin{enumerate}
      \item Architecture and Components
      \item Related Work (Context)
      \item Evaluation and Results
  \end{enumerate}
  \item Conclusions and Future Work
\end{enumerate}
\end{frame}

\section{Motivation and Problem}
\begin{frame}{Problem Illustration}
\begin{figure}
    \centering
    \includegraphics[width=1\linewidth]{Captura de ecrã 2025-12-01 155025.png}
    \label{fig:placeholder}
\end{figure}
\end{frame}

\begin{frame}{Problem: Example Scenario}
	\textbf{Example Timeline}
\begin{itemize}
  \item Jan: Conflict escalation reported (doc A)
  \item Mar: Ceasefire agreement reached (doc B)
  \item Apr Query: ``What is the current status of the War in Ukraine?'' % chktex 38 allow
\end{itemize}
\textbf{Risk:} Retrieves A, ignores B.
\vspace{0.4em} \\
	\textbf{Core Challenge:} Capture and reason over information interdependencies.
\end{frame}

\begin{frame}{Enterprise Information Challenges}
\textbf{Critical Business Context:}
\begin{itemize}
  \item Decisions require accurate, current information
  \item \textbf{High stakes}: compliance, approvals, financial risk
\end{itemize}
\vspace{0.3em}
\textbf{Current Problems:}
\begin{itemize}
  \item Time spent searching and reading
  \item Manual dependency checks = \textbf{high cognitive load}
  \item Reliance on superseded information
  \item \textbf{Modern semantic search fails} to ensure currency/contradictions
\end{itemize}
\end{frame}

\begin{frame}{Context: Edoclink Enterprise}
\textbf{Document Management Platform:}
\begin{itemize}
  \item Workflow-driven lifecycle and business rules
  \item Supports evolution from ad-hoc to complex workflow configurations
  \item End-to-end automation and digitalization
  \item Features:document lifecycle management, collaboration, ERP integration
  \item Deployed in public sector and enterprises
\end{itemize}
\vspace{0.4em}
\end{frame}



\begin{frame}{Edoclink Workflow Illustration}
\begin{figure}
        \centering
  \includegraphics[width=0.75\linewidth]{Images/edockling documents.png}
        \label{fig:placeholder}
\end{figure}
\textbf{Thesis Opportunity:} Leverage this rich structure to enhance semantic search and retrieval accuracy.
\end{frame}


\section{Objectives}

\begin{frame}{Objectives Overview}
\begin{itemize}
  \item \textbf{Represent}: information using Edoclink's structure
  \item \textbf{Detect}: updates and contradictions
  \item \textbf{Enterprise-ready}:
  \begin{itemize}
      \item \textbf{Standardize}: emerging frameworks (MCP)
      \item \textbf{Scale}: large repositories
  \end{itemize}
  \item \textbf{Evaluate}: efficiency and quality
\end{itemize}
\end{frame}

\section{Architecture and Components}
\begin{frame}{Architecture Overview}
Three independent components across the pipeline:
\begin{enumerate}
  \item \textbf{Insertion/Index} (Automatic Knowledge Graph construction)
  \item Query (Query techniques, some that take better advantage of the graph)
  \item Generation (Context given to Agents or LLMs for a readable output)
\end{enumerate}
\vspace{0.3em}
\end{frame}

\begin{frame}{Knowledge Graph Construction}
\begin{enumerate}
  \item Extract entities and cross-references
  \item Merge nodes with matching entities
  \item Store in a graph
\end{enumerate}
\vspace{0.3em}
\centering
\includegraphics[width=0.8\linewidth]{Images/Fluxograma_Data_Processing_Pipeline.jpeg}
\end{frame}

% \begin{frame}{Query Techniques}
% From lowest to highest latency:
% \begin{enumerate}
%   \item Semantic, BM25, and hybrid queries
%   \item Query reformulation (entities/relations) for graph retrieval
%   \item ReAct (Agent) via MCP server <- image
% \end{enumerate}
% \centering
% \includegraphics[width=0.6\linewidth]{Images/Fluxograma_Mini_Query.jpeg}
% \end{frame}

\begin{frame}{Agent Configuration}
\begin{enumerate}
  \item \textbf{Instructions:} Document search assistant connected to Weaviate; deliver concise, sourced answers.
  \item \textbf{Tools:} MCP server translating natural language to schema-valid GraphQL.
\end{enumerate}
\end{frame}

\begin{frame}{Weaviate MCP Server}
\textbf{Context:} Agent–database communication layer \\
\vspace{0.4em}
\textbf{Core Point:} Enables schema-aware, validated queries for agentic retrieval
\begin{itemize}
  \item \textbf{Standard tool interface} for agents
  \item \textbf{Schema validation} for queries
  \item \textbf{Key tools:}
    \begin{itemize}
      \item weaviate-query: hybrid search and direct object retrieval
      \item weaviate-origin: return object with appended references context
      \item weaviate-follow-ref: follow one-hop references and return referenced objects
    \end{itemize}
\end{itemize}
\vspace{0.4em}
\textbf{Conclusion:} Enables agentic graph traversal and reliable, context-grounded answers
\end{frame}

\section{Evaluation and Results}

\begin{frame}{Evaluation Questions}
\begin{itemize}
  \item Does MiniRAG successfully capture document interdependency?
  \item Does the model for retrieval matter?
  \item How much can Agentic system improve retrieval and answer generation?
\end{itemize}
\end{frame}

\begin{frame}{MiniRAG Interdependencies}
    \textbf{Temporal QA:} Time-stamped facts; correctness depends on updates/supersessions. \\
\vspace{0.4em}
    \textbf{Objective:} Test whether MiniRAG profiling + relations improve Token Recall vs naive single-pass RAG. \\
\vspace{0.6em}
	\textbf{Token Recall (Temporal QA Benchmark)}
\begin{center}
\begin{tabular}{lcc}
\hline
System & Benchmark & Thesis (Qwen2.5-3B) \\
\hline
Naive RAG & 43\% & 44\% \\
MiniRAG & 49\% & 38\% \\
MiniRAG (gpt-4o-mini)* & 54\% & -- \\
Multi-Hop (gpt-4o-mini)* & 68.4\% & -- \\
\hline
\end{tabular}
\end{center}
\vspace{0.3em}
\footnote{Benchmarking with larger reasoning models was not performed due to prohibitive computational costs relative to expected benefits.}
\normalsize
\vspace{0.4em}
\end{frame}

% \begin{frame}{RAG generation}
% \textbf{Agentic RAG}: 60.4\% retrieval, 61\% correct answers ($\sim$2--4$\times$ over baselines)
% \vspace{0.3em}
% \begin{center}
%   \begin{table}[htbp]
    \centering
    \begin{tabular}{l r r r}
        \hline
        Approach & Calls & Cost (300 Q) & Retrieval\\
        \hline
        Naive RAG (Qwen2.5) & 1 & 0 & 13.8\% \\
        Naive RAG (GPT-5) & 1 & \$5.07 & 13.4\% \\
        Agentic ReAct (GPT-5) & 2--20 & \$18.35 & 60.4\% \\
        \hline
    \end{tabular}
    \caption{Agentic ReAct used an average of 5 LLM calls per question}
    \label{tab:naive-vs-agentic-tradeoffs}
\end{table}
% \end{center}
% \footnotesize Thresholds via Youden's J statistic.
% \end{frame}

% \begin{frame}{Multi-Collection Retrieval}
% Agentic ReAct benefits from curated collection descriptions that guide semantic queries.
% \begin{itemize}
%   \item More accurate collection selection
%   \item Improvement in agent query reformulation
%   \item More accurate hybrid search
% \end{itemize}
% \begin{center}
%   % Auto-generated metrics table
\begin{table}[t]
\centering
\caption{Multi-collection retrieval and answer quality metrics}
\label{tab:document_retrieved_metrics}
\begin{tabular}{lrr}
\hline
Method & Avg. Retrieval Rate & Correct Answers \\
\hline
Mixed REST & 46.5\% & 15.8\% \\
Mixed LexRank & 39.4\% & 24.6\% \\
Mixed DOCX & 33.0\% & 31.5\% \\
Naive RAG & 13.8\% & 19.1\% \\
\hline
\end{tabular}
\end{table}

% \end{center}
% \end{frame}

% \begin{frame}{Summarization \& Storage}
% Selecting relevant text reduces noise and improves retrieval.
% \begin{table}[t]
% \centering
% \caption{Document retrieval and summarization metrics}
% \label{tab:document_retrieved_metrics_summarization}
% \begin{tabular}{lrrr}
% \hline
% Approach & Retrieval & Token Recall & Jaccard \\
% \hline
% Agentic ReAct (GPT-5)& 60.4\% & 61.1\%  & 55.7\%\\
% Naive RAG (LexRank) & 33.6\% & 29.5\%  & 35.9\% \\
% Naive RAG (BART) & 17.8\% & 3.0\% & 11.1\% \\
% Naive RAG (Qwen2.5) & 13.8\% & 19.1\% & 22.1\% \\
% \hline
% \end{tabular}
% \end{table}
% \footnotesize Thresholds via Youden's J statistic.
% \end{frame}

\begin{frame}{Edoclink Integration}
\begin{itemize}
  \item Use Weaviate cross-references in a workflow-organized company database.
  \item Enforce ingestion rules for consistency and provenance.
  \item Agent performs multi-hop queries: files $\rightarrow$ stages $\rightarrow$ flows $\rightarrow$ entities.
  \item Store information in focused snippets for robust retrieval within context limits.
\end{itemize}
% \begin{figure}
%   \centering
%   \includegraphics[width=0.6\linewidth]{image.png}
%   \caption{Agent path over cross-referenced workflow to answer a question. "How is the war on Russia and Ukraine?"}
%   \label{fig:workflow_agent_path}
% \end{figure}
\end{frame}
 \begin{frame}{Structured Reasoning}
     \begin{figure}
         \centering
         \includegraphics[width=0.9\linewidth]{figures/graphviz.png}
         \label{fig:placeholder}
     \end{figure}
 \end{frame}

\begin{frame}{Final Answer}
    Final answer. It remains an active, violent conflict. \\
    Grounding (evidence). \\
    • A matched stage in the “Russia vs Ukraine War” flow is tied to a report dated 2024-07-12 titled 
    “Drone attack hits military base”. \\
    • Excerpt: “A drone strike has reportedly destroyed parts of a military base near the front. Witnesses 
    describe significant damage and casualties.” \\
    \textbf{ Interpretation.}
    • Continued strikes on military targets near the front and ongoing casualties, involving both Russia
    and Ukraine as the referenced entities
\end{frame}

\begin{frame}{Final Answer}
    - Final output: \\
    Short answer: Fighting is ongoing. \\

    Grounding from our records: \\
    - 2024-07-12: “A drone strike has reportedly destroyed parts of a military base near the front. Witnesses describe significant damage and casualties.” (Drone attack hits military base)
    - 2024-07-10: “A peace treaty has been signed between Russia and Ukraine, ending hostilities.” (Peace treaty signed) \\

    Given the later 2024-07-12 report of a strike, hostilities appear to have continued despite the earlier treaty claim.
\end{frame}
\section{Related Work}
\begin{frame}{Related Work}
\begin{itemize}
  \item Google NotebookLM: personal, source-grounded assistant over curated notes; footnoted syntheses and study guides
  \item Perplexity AI: web-grounded conversational search with inline citations and multi-document aggregation
  \item Vector databases : Weaviate, Qdrant, and Milvus.
 Put slides in table
\end{itemize}
\end{frame}

\begin{frame}{Comparison to Prior Approaches}
Multi-hop Agentic RAG for workflow-structured enterprise content (e.g., Edoclink)
\begin{itemize}
  \item Integrates schema-driven traversal
  \item Standardizes graph access (MCP server)
  \item Focus: enterprise interdependency reliability
\end{itemize}
\end{frame}

\section{Conclusions and Future Work}

\begin{frame}{Conclusions}
\begin{itemize}
  \item \textbf{Graph-aware reasoning}: Moderately prevents outdated answer\pause
  \item \textbf{Structure-aware retrieval}: leverages cross-references for reasoned traversal\pause % chktex 1 allow
  \item \textbf{Well-instructed agents}: schema-aware tools guide retrieval and sourcing\pause % chktex 1 allow
\end{itemize}
\end{frame}

\begin{frame}{Limitations}
\begin{itemize}
  \item Scale constrained (compute / budget)
  \item No public benchmark for workflow organized information (construction overhead)
  \item NLP ecosystem is highly dynamic; conclusions are time-bound % chktex 1 allow
  \item Model and embedding churn can change retrieval behavior % chktex 1 allow
  \item RAG and agent frameworks evolve; pin versions, datasets, prompts, and eval protocols % chktex 1 allow
\end{itemize}
\end{frame}

\begin{frame}{Future Work}
\begin{itemize}
  \item Integrate with Edoclink, including its access controls
  \item Beta-test multi-hop Agentic RAG in production
  \item Release structure-aware retrieval benchmark
  \item Add caching to reduce multi-hop latency
  \item Agentic workflows: Graph RAG, outdated info handling, memory
  \item Additional MCP tools for adjacent applications
\end{itemize}
\end{frame}

\begin{frame}{Thank You}
\begin{center}
  \begin{block}
    {\LARGE Thank you!}
  \end{block}
  \vspace{1em}
  Francisco Azeredo\\
  Instituto Superior T\'ecnico\\
  Questions?
\end{center}
\end{frame}

\end{document}
%%%%%%%%%%%%%%%%%%%%%%%%%%%%%%%%%%%%%%%%%%%%%%%%%%%%%%%%%%%%%%%%%%%%%%

